\documentclass[10pt,nofootinbib]{revtex4}
\usepackage[nocap]{ctex}

%\usepackage{xeCJK}
%\setCJKmainfont{Source Han Sans CN}
%\setCJKmonofont{Source Han Sans CN}
%\setCJKsansfont{Source Han Sans CN}

\usepackage{amsmath,amssymb,amsfonts,mathrsfs,bm,dsfont}
\usepackage{enumerate}
\usepackage{enumitem} % Customize itemize, see https://ctan.org/tex-archive/macros/latex/contrib/enumitem/
\usepackage[all]{xy}
\usepackage[normalem]{ulem}	% delete line
\usepackage{graphics,color}
\usepackage{tikz}
	\usetikzlibrary{calc}
	\usetikzlibrary{decorations.markings}
	\usetikzlibrary{arrows}
	\usetikzlibrary{patterns}
	%\usetikzlibrary{shapes.callouts}
\tikzset{
    level/.style = {
        ultra thick,
        blue,
    },
    connect/.style = {
        dashed,
        red
    },
    label/.style = {
        text width=2cm
    }
}
\usepackage{pgfplots}
%\usepackage[citestyle=authortitle]{biblatex} % able to cite the title, author and year
%\usepackage{hyperref}
\usepackage{feynmp} % feymann diagram
\usepackage{extarrows}

\usepackage[normalem]{ulem} % 文字划掉(横),与 cite 兼容问题,见 https://tex.stackexchange.com/questions/98222/ulem-incompatibility-with-multiple-entries-in-cite

\newcommand*\dd{\mathop{}\!\mathrm{d}}
\newcounter{Claim}[section]
\newenvironment{Claim}[1][]{{\par\normalfont\bfseries \underline{Claim~\stepcounter{Claim}\arabic{Claim}.}~#1~~}}{\par}
\newcounter{Proposition}[section]
\newenvironment{Proposition}[1][]{{\par\normalfont\bfseries \underline{Proposition~\stepcounter{Proposition}\arabic{Proposition}.}~#1~~}}{\par}
\newcounter{Note}[section]
\newenvironment{Note}[1][]{{\par\normalfont\bfseries \underline{Note~\stepcounter{Note}\arabic{Note}.}~#1~~}}{\par}
\newcounter{Lemma}[section]
\newenvironment{Lemma}[1][]{{\par\normalfont\bfseries \underline{Lemma~\stepcounter{Lemma}\arabic{Lemma}.}~#1~~}}{\par}
\newcounter{Corollary}[section]
\newenvironment{Corollary}[1][]{{\par\normalfont\bfseries \underline{Corollary~\stepcounter{Corollary}\arabic{Corollary}.}~#1~~}}{\par}
\newenvironment{Proof}{{\par~{\normalfont\bfseries $\vartriangleright$}~~}}{\hfill $\square$\par\hfill\par} %\par
\newcounter{Def}[section]
\newenvironment{Def}[1][]{{\par\normalfont\bfseries \underline{Definition~\stepcounter{Def}\arabic{Def}.}~#1~~}}{\par}

\allowdisplaybreaks[4] %允许 align 跨页编排


%\def\checkmark{\tikz\fill[scale=0.4](0,.35) -- (.25,0) -- (1,.7) -- (.25,.15) -- cycle;}
%\def\G{\mathcal{G}}
\def\Z{\mathcal{Z}}
\def\H{\mathcal{H}}

\begin{document}
\title{Hydrodynamic Methods in Condensed Matter}
\author{Xiaodong Hu}
%\altaffiliation[Also at ]{Boson College}
\email{xiaodong.hu@bc.edu}
\affiliation{Department of Physics, Boston College}

\date{\today}

\begin{abstract}
	In this note I will review modern methods of hydrodynamics in condensed matter systems (unlike old one like superfluids and spin waves with orderes phase), from quantum critical Nernst effects to fractional quantum Hall system \par
	%\begin{center}
		\hfill\par
		{\centering\kaishu 云行雨施,品物流形。\\[0.5em]}
	%\end{center}
	\hfill------ 「易经·彖」
\end{abstract}

\maketitle
\tableofcontents

\section{Hydrodynamics in Symmetry-breaking Phase --- Spin Waves}
	\subsection{$S=1/2$ XXZ Model and Conservation Law}
		We will consider $S=1/2$ XXZ model in a uniform magnetic field $H_z$ in this section
		\begin{equation}\label{1.1.1}
			\H=\H_{XXZ}-H_z\sum_i S_i^z\equiv-\sum_{\langle ij\rangle}J_\perp(S_i^xS_j^x+S_i^yS_j^y)-\sum_{\langle ij\rangle}J_zS_i^zS_j^z-H_z\sum_i S_i^z.
		\end{equation}
		Whatever the phase is (\emph{planar ferromagnetic} $J_\perp>|J_z|>0$, \emph{planar antiferromagnetic} $J_\perp<-|J_z|<0$, or \emph{planar paramagnetic} $T>T_c$ when $M_\perp e^{i\varphi}\equiv\langle S_i^x\rangle+i\langle S_i^y\rangle$ vanishes \cite{halperin1969hydrodynamic,chaikin2000principles}), Hamiltonian \eqref{1.1.1} is clearly invariant under the global spin rotation along $\hat{S}^z$ axis. So we always have a conservation of $S^z$ operator, or coarse-grained $z$-component magnetization $M_z\equiv\langle S_i^z\rangle$ under ensemble average, namely
		\begin{equation}\label{1.1.2}
			\partial_t m_z+\nabla\cdot\bm{j}^{m_z}=0,
		\end{equation}
		where we introduce the intensive magnetization density $m_z$ and corresponding spin current $\bm{j}^{m_z}$. Apparently total energy density $E\equiv\langle \H_{XXZ}\rangle$, or energy density $\varepsilon$ is also conserved, giving
		\begin{equation}\label{1.1.3}
			\partial_t \varepsilon+\nabla\cdot\bm{j}^\varepsilon=0.
		\end{equation}
		\indent But given $M_z$ and $E$, one still cannot determine the nonuniform states out of equilibrium like ferromagnetic phase below $T_c$. In fact, one must also specify the direction of alignment of perpendicular magnetization $\varphi$, which is locally defined as
		\begin{equation*}
			m_x(\bm{r})+im_y(\bm{r})=m_\perp(\bm{r})e^{i\varphi(\bm{r})}.
		\end{equation*}
		\indent {\bf In hydrodynamic regime, relevant hydro-variables are those whose long-wavelength variations vary slowly in time compared with the characteristic microscopic relaxation time of the scattering processes}. In symmetry breaking phase, such variables are of two classes:
		\begin{itemize}
			\item densities of conserved variables (here is $m_z(\bm{r})$ and $\varepsilon(\bm{r})$);
			\item symmetry-breaking elastic variables (here is $\varphi(\bm{r})$).
		\end{itemize}
		Thus we \emph{assume} that {\bf each point of the system reaches almost thermodynamic equilibrium at each instant of time, so that the system is \emph{completely} determined by the hydro-variables, even though they vary in time}. Particularly, the above undefined current $\bm{j}^e$ and $\bm{j}^{m_z}$ should be functional of these hydro-variables. Ditto for the symmetry-breaking field if we introduce a scalar valued functional $\psi$ of hydro-variables
		\begin{equation}\label{1.1.4}
			\partial_t \varphi(\bm{r})+\psi(\bm{r})=0.
		\end{equation}
		Noting that it is the fluctuation of $\nabla\varphi$ rather than $\varphi$ that contributes to physical properties such as entropy like $\varepsilon$ and $m_z$. So they are in the same order of fluctuation and we will take $\bm{v}(\bm{r})\equiv\nabla\varphi(\bm{r})$ as a replacement of fundamental hydro-variables ($\bm{v}$ is in anolagous of superfluid velocity $\bm{v}_s$ in liquid helium), satisfying
		\begin{equation}\label{1.1.5}
			\partial_t\bm{v}(\bm{r})+\nabla\psi(\bm{r})=0.
		\end{equation}
		Equation \eqref{1.1.2}, \eqref{1.1.3}, and \eqref{1.1.5} consist of the complete conservation law of our system.

	\subsection{Constitutive Relations}
		Equations relating coarse-grained current with hydro-variables are called \emph{constitutive relations}. Introducing the conjugate field $\bm{x}$ of $\bm{v}$, and $h$ of $m_z$, the first law of thermodynamics (in true thermodynamic eqiuilibrium) tells us
		\begin{equation}\label{2.0.1}
			T\dd s\equiv\dd \varepsilon-h\dd m_z-\bm{x}\cdot\dd\bm{v},
		\end{equation}
		where $h\equiv-T \partial s/\partial m_z|_S$ and $\bm{x}\equiv-T\partial s/\partial \bm{v}|_S$. 

		\subsubsection{Zeroth Order (Non-dissipative)}
			In this subsection, we will {\bf use (assume) the conservation of local entropy to constrain the \emph{non-dissipative} part of ``current operators'' defined above}. \par
			To the zeroth order of fluctuations, i.e., spatial derivatives of hydro-variables, the most general form of current operator we can write is
			\begin{equation}\label{2.1.1}
				\bm{j}^\varepsilon\equiv A(\bm{r})\bm{v}(\bm{r}),\quad \bm{j}^{m_z}\equiv B(\bm{r})\bm{v}(\bm{r}),\quad \psi\equiv C(\bm{r}),
			\end{equation}
			where coefficients $A,B$ and $C$ are scalar functional of $\varepsilon(\bm{r})$ and $m_z(\bm{r})$ waiting to be determined. In general, coefficients $A$ and $B$ may be chosen as rank-$2$ tensors of functionals of $\varepsilon$ and $m_z$, such that the two vector-valued current operators $\bm{j}^\varepsilon$ and $\bm{j}^{m_z}$ are not in the same direction as $\bm{v}$. {\color{red}But here we shall assume lattice \emph{cubic symmetry} \cite{halperin1969hydrodynamic} to forbid such possibility}. Simiarly, one can contract vector $\bm{v}$ with a rank-2 tensor to form a scalar legal in the expression of $\psi$. {\color{red}But we shall assume \emph{reflection symmetry} to forbid such terms for convenience}.\par

			Fortunately, as we will see below, $C(\bm{r})$ can be obtained in advance without entropy argument \cite{chaikin2000principles,halperin1969hydrodynamic}.\par
			Given $\varepsilon,m_z$ and $\bm{v}=0$, we know from \eqref{1.1.5} that the equilibrium state \emph{without} external magnetic field allows a constant precession rate $\dd\varphi/\dd t=-\psi_0(\varepsilon,m_z)$. Let us prepare such stationary state and turn on $H_z$. The time evolution of $M_\perp$ is now dominated by the Hamiltonian $\mathcal{H}\equiv\mathcal{H}_{XXZ}+\mathcal{H}_{ext}$. Since $\mathcal{H}_{ext}\equiv-H_z\sum_i S_i^z$ commutes with $\mathcal{H}_{XXZ}$, time evolution of $m_\perp(t)\equiv\langle S_i^+(t)\rangle$ gives
			\begin{equation*}
				\dfrac{\dd\varphi}{\dd t}=-(\psi_0(\varepsilon,m)+H_z).
			\end{equation*}
			However, in \emph{true} thermodynamic equilibrium (with non-vanishing $H_z$), where the system energy $E-M_z H_z$ is minimized so that
			\begin{equation}\label{2.1.2}
				\dfrac{\partial E}{\partial M_z}=\dfrac{\partial \varepsilon}{\partial m_z}=H_z,
			\end{equation}
			the procession rate $\dd\varphi/\dd t$ must be zero, otherwise the rotating $m_\perp$ will radiate and lose energy \cite{halperin1969hydrodynamic}. Thus $\psi_0$ coincides with $H_z$ in \emph{true} equilibrium. On the other hand, the fisrt law of thermodynamics \eqref{2.0.1} tells us the left hand side of \eqref{2.1.2} is nothing but conjugate field $h(\varepsilon,m_z)\equiv\frac{\partial \varepsilon}{\partial m_z}$. Therefore, we conclude for general \emph{non-dissipative} stationary state 
			\begin{equation}\label{2.1.3}
				\dfrac{\dd\varphi}{\dd t}=-(h-H_z),
			\end{equation}
			or
			\begin{equation}\label{2.1.4}
				C(\bm{r})=H_z-h(\varepsilon(\bm{r}),m_z(\bm{r})).
			\end{equation}
			\indent The left work is to determine $A(\bm{r})$ and $B(\bm{r})$ in \eqref{2.1.1} with the help of entropy conservation (at zeroth order).\par
			Hydrodynamic assumption ensures local entropy density to be the functional of hydro-variables $s=s(\varepsilon,m_z,\bm{v})$. And at each slice of time, each point of the system is assumed to reach local thermodynamic eqiuilibrium. Therefore we can expand entropy density around its equilibrium value $s_0$ (which must be the maximum of $s$ due to the second law of thermodynamics) that 
			\begin{equation}\label{2.1.5}
				s\simeq s_0(\varepsilon)-\dfrac{1}{2T}\chi_s^{-1}m_z^2-\dfrac{\rho_s}{2T}v^2,
			\end{equation}
			where temperature $T^{-1}\equiv\partial s_0/\partial\varepsilon$ is well-defined in equilibrium and inserted by convention. Magnetic suseptibility $\chi_s$ and the magnetic version of ``superfluid density'' $\rho_s$ may also be functional of hydro-variables, but up to the second order of disturbance, we can safely treat them as constants\footnote{Apparently $\chi_s$ and $\rho_s$ must be nonnegative for stability of the system.}. Therefore the two conjugate fields we introduce above takes the form of $h\equiv-T \partial s/\partial m_z=\chi_s^{-1}m_z$ and $\bm{x}\equiv-T\partial s/\partial \bm{v}=\rho_s\bm{v}$.\par
			Taking time derivative of the first law of thermodynamics and substituting conservation laws \eqref{1.1.3}, \eqref{1.1.4}, \eqref{1.1.5} and the zeroth order of non-dissipative currents \eqref{2.1.1} and \eqref{2.1.4}, we have
			\begin{align}\label{2.1.6}
				T\dfrac{\partial s}{\partial t}&=\dfrac{\partial \varepsilon}{\partial t}-h\dfrac{\partial m_z}{\partial t}-\bm{x}\cdot\dfrac{\partial \bm{v}}{\partial t}=-\nabla\cdot\bm{j}^\varepsilon+h\nabla\cdot\bm{j}^{m_z}-\bm{x}\cdot\nabla(h-H_z)\nonumber\\
				&=-\nabla\cdot(A\bm{v})+h\nabla\cdot(B\bm{v})-\bm{x}\cdot\nabla h\nonumber\\
				&=-\nabla\cdot\bigg((A-Bh)\bm{v}\bigg)-(B\bm{v}+\bm{x})\cdot\nabla h.
			\end{align}
			Writing the \emph{heat current} $\bm{Q}\equiv (A-Bh)\bm{v}$ and using the identity
			\begin{equation*}
				\nabla\cdot\bm{Q}\equiv T\nabla\cdot \left(\dfrac{\bm{Q}}{T}\right)+\bm{Q}\cdot \left(\dfrac{\nabla T}{T}\right),  
			\end{equation*}
			Equation \eqref{2.1.6} becomes
			\begin{equation}\label{2.1.7}
				T\dfrac{\dd s}{\dd t}\equiv T \left(\dfrac{\partial s}{\partial t}+\nabla\cdot \left(\dfrac{\bm{Q}}{T}\right) \right)=-\bm{Q}\cdot \left(\dfrac{\nabla T}{T}\right)-(B\bm{v}+\bm{x})\cdot\nabla h
			\end{equation}
			if we identify $\bm{Q}$ as the \emph{entropy current density}. In the absence of dissipation, as is assumed to the zeroth order disturbance, entropy is conserved. Thus
			\begin{equation}\label{2.1.8}
				\begin{cases}
					\bm{Q}\equiv (A-Bh)\bm{v}=0\\
					B\bm{v}+\bm{x}=0
				\end{cases}\implies
				\begin{cases}
					A\bm{v}=-\rho_s\bm{x}\\
					B\bm{v}=-\bm{x}
				\end{cases}.
			\end{equation}
			Inserting the expression of $h$ and $\bm{v}$, finally we come to the \emph{non-dissipative} linearized equation of motion up to the zeroth order of gradients of hydro-variables
			\begin{align}
				\partial_t\varepsilon&=\dfrac{\rho_s}{\chi_s}\nabla\cdot(m_z\bm{v})\label{2.1.9}\\
				\partial_t m_z&=\rho_s\nabla\cdot\bm{v}\label{2.1.10}\\
				\partial_t \bm{v}&=\dfrac{1}{\chi_s}\nabla m_z\label{2.1.11}.
			\end{align}
			Taking the time derivative of \eqref{2.1.10} and substituting \eqref{2.1.11}, one immediately have
			\begin{equation}\label{2.1.12}
				\dfrac{\partial^2 m_z}{\partial t^2}=-\dfrac{\rho_s}{\chi_s}\nabla^2m_z,
			\end{equation}
			which predicts a \emph{undamped spin-wave mode} $\omega=\pm ck$ with $c\equiv(\rho_s/\chi_s)^{1/2}$.

		\subsubsection{First Order (Dissipative)}
			To the first order of gradients of hydro-variables, howevere, we will use {\bf the positivity of local entropy production} to constrain further the \emph{dissipative} part of currents
			\begin{equation}\label{2.2.1}
				\bm{j}^\varepsilon\equiv-\rho_s\bm{x}+{\bm{j}^\varepsilon}',\quad\bm{j}^{m_z}\equiv-\bm{x}+{\bm{j}^{m_z}}',\quad \psi\equiv H_z-h+\psi'.
			\end{equation}
			Here the zeroth-order result has been used.	Similar to the precedure we have done above, introducing the first-order \emph{heat current} $\bm{Q'}\equiv{\bm{j}^\varepsilon}'-h{\bm{j}^{m_z}}'+\rho_s\bm{v}\psi'$, the local entropy production now takes the form of
			\begin{equation}\label{2.2.2}
				T\dfrac{\dd s}{\dd t}\equiv T \left(\dfrac{\partial s}{\partial t}+\nabla\cdot\bm{Q'}\right)=-\bm{Q'}\cdot\left(\dfrac{\nabla T}{T}\right)-{\bm{j}^{m_z}}'\cdot\nabla h-\psi'\nabla\cdot\bm{x}.
			\end{equation}
			The most general constitutive relation up to the \emph{first order} we can construct are
			\begin{align*}
				\bm{Q}'&=-K_{11}\nabla T-K_{12}\nabla h\\
				{\bm{j}^{m_z}}&=-K_{21}\nabla T-K_{22}\nabla h\\
				\psi'&=K_{33}\rho_s\nabla\cdot\bm{v}.
			\end{align*}
			In general both $\bm{Q}'$ and ${\bm{j}^{m_z}}'$ may contain terms proportional to gradients of $\bm{v}$ like $\nabla\times\bm{v}$. But definition $\bm{v}\equiv\nabla\varphi$ forbids such terms. Also, {\color{red}\bf because \emph{dissipative} motion is related to \emph{irreversible} thermodynamic processes, dissipative currents of one sign under time reversal must be proportional to variables of the opposite sign}. Namely, off-diagonal couplings $K_{12}$ and $K_{21}$ must be zero as well. So we are left with
			\begin{equation}\label{2.2.3}
				\bm{Q}'=-\kappa\nabla T,\quad {\bm{j}^{m_z}}'=-K_{22}\nabla h,\quad \psi'=-K_{33}\rho_s\nabla\cdot\bm{v},
			\end{equation}
			where we interprete $K_{11}$ as the \emph{thermal conductivity} $\kappa$.\par
			To make equations \eqref{2.2.3} close, let us restrict ourselves to the case where magnetization $M_z=0$. This is appropriate for fluctuations about true thermodynamic equilibrium without external magnetic fields. In this case, we have
			\begin{equation*}
				\dd T=C^{-1}\dd\varepsilon
			\end{equation*}
			if we introduce specific heat $C$. Inserting back the expression of dissipative currents to conservation laws, we come to the \emph{dissipative} linearized\footnote{By linear, we mean equation of motion is expanded to the first power of hydro-variables.} equation of motion up to the first-order fluctuation of hydro-variables
			\begin{align}
				\partial_t \varepsilon&=\nabla\cdot\bigg(\kappa\nabla T+K_{22}h\nabla h-K_{33}\rho_s^2\bm{v}\nabla\cdot\bm{v}\bigg)\simeq\kappa C^{-1}\nabla^2\varepsilon\label{2.2.4}\\
				\partial_t m_z&=\rho_s\nabla\cdot\bm{v}+\chi_s^{-1}K_{22}\nabla^2m_z\label{2.2.5}\\
				\partial_t\bm{v}&=\chi_s^{-1}\nabla m_z+K_{33}\rho_s\nabla(\nabla\cdot\bm{v})\label{2.2.6}.
			\end{align}
			Clearly solution of \eqref{2.2.4} yields a heat \emph{diffusion mode}
			\begin{equation}\label{2.2.7}
				\omega_\varepsilon(\bm{k})=-i\kappa C^{-1}k^2=:-iD_Tk^2.
			\end{equation}
			Another \emph{damped spin-wave mode} from coupled equations \eqref{2.2.5} and \eqref{2.2.6} can be obtained by diagnolizing the coefficients matrix in momentum space and expanding up to $k^2$, giving
			\begin{equation}\label{2.2.8}
				\omega_{\pm}(\bm{k})=\pm ck-\dfrac{1}{2}iDk^2,
			\end{equation}
	\subsection{Linear Response}

\section{Transport Near Quantum Criticality}
	In this section, we focus on \emph{Lorentz-invariant} quantum critical points in the \emph{hydrodynamics region} that external electromagnetic frequencies satisfy $\hbar\omega\ll k_B T$. This condition is widely (actually, almost all) performed in experiments but mismatch the assumption to many theoretical calculation as well as numerical simulation in, for example, DC conductivity near superfluid-insulator phase transition \cite{damle1997nonzero}.\par
	\textbf{The fundamental ingredients of hydrodynamic analysis are the conserved quantities and their equations of motion}. But before preceeding, let me pause and comment on the old hydrodyamics of Lifshitz superfluids and spin-waves \cite{hohenberg1977theory,halperin1969hydrodynamic}. {\color{red} In hydrodynamics region, $k_BT/\hbar$ naturally plays the role of low-energy frequency characterizing the relation time that we are interested in, while in old prescription the relaxation time just diverges (so the energy scale may not be correct) \cite{hartnoll2007theory}}.

	\subsection{General Setup}
		Our system subject to external magnetic field for transport study. Invariance of \emph{gauge transformation} $A_\mu\mapsto A_\mu+\partial_\mu f$ and \emph{diffeomorphism transformation} $x^\mu\mapsto x_\mu+\xi_\mu$ of action gives\footnote{See, for example, section II D of \cite{herzog2009lectures}. The result can also be proved holographically, see \cite{lindgren2015holographic}.}
		\begin{align}\label{5.1.1}
			\partial_\mu J^\mu&=0,\\
			\partial_\mu T^{\mu\nu}&=F^{\mu\nu}J_\mu.
		\end{align}
		respectively.
		
	\subsection{Hydrodynamic Modes and Linear Response}
	\subsection{Perspetive from Holography}

\section{Non-relativistic Transport --- Fractional Quantum Hall System}
	\subsection{Spacetime Background}
	\subsection{Response}


\bibliography{hxd}
\bibliographystyle{apsrev} % apsrev is format for PRL of APS
\end{document}