\documentclass[10pt,nofootinbib,letterpaper]{revtex4}
\usepackage[nocap]{ctex}

%\usepackage{xeCJK}
%\setCJKmainfont{Source Han Sans CN}
%\setCJKmonofont{Source Han Sans CN}
%\setCJKsansfont{Source Han Sans CN}

\usepackage{amsmath,amssymb,amsfonts,mathrsfs,bm,dsfont}
\usepackage{enumerate}
\usepackage{enumitem} % Customize itemize, see https://ctan.org/tex-archive/macros/latex/contrib/enumitem/
\usepackage[all]{xy}
\usepackage[normalem]{ulem}	% delete line
\usepackage{graphics,color}
\usepackage{tikz}
	\usetikzlibrary{calc}
	\usetikzlibrary{decorations.markings}
	\usetikzlibrary{arrows}
	\usetikzlibrary{patterns}
	%\usetikzlibrary{shapes.callouts}
\tikzset{
    level/.style = {
        ultra thick,
        blue,
    },
    connect/.style = {
        dashed,
        red
    },
    label/.style = {
        text width=2cm
    }
}
\usepackage{pgfplots}
%\usepackage[citestyle=authortitle]{biblatex} % able to cite the title, author and year
%\usepackage{hyperref}
\usepackage{feynmp} % feymann diagram
\usepackage{extarrows}

\usepackage[normalem]{ulem} % 文字划掉(横),与 cite 兼容问题,见 https://tex.stackexchange.com/questions/98222/ulem-incompatibility-with-multiple-entries-in-cite

\newcommand*\dd{\mathop{}\!\mathrm{d}}
\newcounter{Claim}[section]
\newenvironment{Claim}[1][]{{\par\normalfont\bfseries \underline{Claim~\stepcounter{Claim}\arabic{Claim}.}~#1~~}}{\par}
\newcounter{Proposition}[section]
\newenvironment{Proposition}[1][]{{\par\normalfont\bfseries \underline{Proposition~\stepcounter{Proposition}\arabic{Proposition}.}~#1~~}}{\par}
\newcounter{Note}[section]
\newenvironment{Note}[1][]{{\par\normalfont\bfseries \underline{Note~\stepcounter{Note}\arabic{Note}.}~#1~~}}{\par}
\newcounter{Lemma}[section]
\newenvironment{Lemma}[1][]{{\par\normalfont\bfseries \underline{Lemma~\stepcounter{Lemma}\arabic{Lemma}.}~#1~~}}{\par}
\newcounter{Corollary}[section]
\newenvironment{Corollary}[1][]{{\par\normalfont\bfseries \underline{Corollary~\stepcounter{Corollary}\arabic{Corollary}.}~#1~~}}{\par}
\newenvironment{Proof}{{\par~{\normalfont\bfseries $\vartriangleright$}~~}}{\hfill $\square$\par\hfill\par} %\par
\newcounter{Def}[section]
\newenvironment{Def}[1][]{{\par\normalfont\bfseries \underline{Definition~\stepcounter{Def}\arabic{Def}.}~#1~~}}{\par}
\newcounter{Assumption}[section]
\newenvironment{Assumption}[1][]{{\par\normalfont\bfseries \underline{Assumption~\stepcounter{Assumption}\arabic{Assumption}.}~#1~~}}{\par}



\allowdisplaybreaks[4] %允许 align 跨页编排


%\def\checkmark{\tikz\fill[scale=0.4](0,.35) -- (.25,0) -- (1,.7) -- (.25,.15) -- cycle;}
%\def\G{\mathcal{G}}
\def\Z{\mathcal{Z}}
\def\H{\mathcal{H}}

\begin{document}
\title{Hydrodyanmic EOM from Boltzmann Equation}
\author{Xiaodong Hu}
%\altaffiliation[Also at ]{Boson College}
\email{xiaodong.hu@bc.edu}
\affiliation{Department of Physics, Boston College}

\date{\today}

\begin{abstract}
	In this note I will first introduce the correction of phase space volume element from Berry curvature term in semi-classic EOM, then combine it with Boltzmann Kinetic equation to obtain both the constitution relation and hydrodynamic equation of motion.\par
	%\begin{center}
		\hfill\par
		{\centering\kaishu 流成笔下春风瓣,吹散弦上秋草声。\\[0.5em]}
	%\end{center}
	\hfill------ 雨楼清歌「一瓣河川」
\end{abstract}

\maketitle
\tableofcontents

\section{Correction of Phase Space Volume Element}
	\subsection{Symplectic Manifold}
		Let us first recall the \emph{symplectic structure} of Hamiltonian dynamics. Physically, a ($6N$-dimensional) phase space is a \emph{symplectic manifold} $(M,\omega)$ consisting of a smooth manifold $M$ and a closed non-degenerate\footnote{By close we mean $\dd\omega\equiv0$, and by non-degenerate we mean for any $\bm{v}$, $\omega(\bm{u},\bm{v})=0\implies \bm{u}=\bm{0}$.} differential $2$-form $\omega$, which in general takes the form of
		\begin{equation*}
			\omega=\dfrac{1}{2}\omega_{\alpha \beta}\dd\xi^\alpha\wedge\dd\xi^\beta
		\end{equation*}
		in local coordinates $\{\xi^a\}$. Given a simplectic space $V\in T_pM$, we can always define a linear map $\omega^\sharp:V\rightarrow V^*$ by $\bm{v}\mapsto\omega(\cdot,\bm{v})$ since $\omega$ is bilinear. Clearly simplectic $2$-form $\omega$ is non-degenerate iff $\omega^\sharp$ is one-to-one\footnote{For a linear map $\omega^\sharp$, one-to-one is equivalent to $\mathrm{Ker}(\omega^\sharp)=\{\bm{0}\}$, namely $\omega^\sharp(\bm{u})=\bm{0}\implies\bm{u}=\bm{0}$. But $\omega^\sharp(\bm{u})=\bm{0}\implies \forall \bm{v}\in V, \omega^\sharp(\bm{u})(\bm{v})\equiv\omega(\bm{u},\bm{v})=0\implies\bm{u}=\bm{0}$ by the non-degeneracy of $\omega$.}, so $\omega^\sharp$ is an isomorphism and has an inverse $(\omega^\sharp)^{-1}$.\par
		Equipped with $\omega^\sharp$, we can then define the \emph{Hamiltonian vector field} $X_f$ for any differential function $f\in C^\infty(M)$ by $X_f:=(\omega^\sharp)^{-1}\dd f$. Specifically, in the local coordinates $\{\xi^\alpha\}$, we have
		\begin{equation*}
			\omega^\sharp\left(\dfrac{\partial }{\partial \xi^\gamma}\right)\equiv\iota_{\frac{\partial }{\partial \xi^\gamma}}\omega\equiv\dfrac{1}{2}\omega^{\alpha\beta}\dd\xi^\alpha\wedge\dd\xi^\beta\left(\dfrac{\partial }{\partial \xi^\gamma}\right)=\omega_{\gamma\beta}\dd\xi^\beta\implies (\omega^\sharp)^{-1}(\dd\xi^\alpha)=\omega^{\alpha\gamma}\dfrac{\partial }{\partial \xi^\gamma}
		\end{equation*}
		(where we introduce the inverse matrix $\omega^{\alpha\gamma}$ such that $\omega^{\alpha\gamma}\omega_{\gamma\beta}\equiv\delta^\alpha_\beta$) and so
		\begin{equation*}
			X_f=\partial_\alpha f(\omega^\sharp)^{-1}(\dd\xi^\alpha)=\omega^{\alpha\gamma}\partial_\alpha f\dfrac{\partial }{\partial \xi^\gamma}.
		\end{equation*}
		Then Poisson bracket for two function $f,g\in C^\infty(M)$ can be defined through the two corresponding Hamiltonian vector fields
		\begin{equation}\label{1.1.1}
			{\color{red}\{f,g\}:=\omega(X_f,X_g).}
		\end{equation}
		In the local coordinates, we immediately have
		\begin{equation}\label{1.1.2}
			\{f,g\}=\omega^{\alpha \beta}\partial_\alpha f\partial_\beta g.
		\end{equation}
		\indent In QM, the eigenvalues of a Hamiltonian operator serves as a differential function of local coordinates $h(\xi)\in C^\infty(M)$, so the semi-classical \emph{Hamiltonian equation} (as the second ingredients of Hamiltonian dynamics) reads
		\begin{equation}\label{1.1.3}
			{\color{red}\dot{\xi}^\alpha=\{h,\xi^\alpha\}=\omega^{\alpha\beta}\partial_\beta h,\quad \text{or}\quad\omega_{\alpha\beta}\dot{\xi}^\beta=\partial_\alpha h}.
		\end{equation}

	\subsection{Hamiltonian Dynamics with Berry Curvature}
		Semi-classical EOM of electronic wave-packets reads
		\begin{align}
			\dot{\bm{r}}&=\dfrac{1}{\hbar}\dfrac{\partial \varepsilon_n}{\partial \bm{k}}-\dot{\bm{k}}\times\Omega(\bm{k}),\label{1.2.1}\\
			\hbar\dot{\bm{k}}&=-e\bm{E}-e\dot{\bm{r}}\times\bm{B}.\label{1.2.2}
		\end{align}
		In the presence of both electric magnetic fields, the Hamiltonian function of such quasiparticle is $h=\varepsilon_n-eV$. We can re-arrange the kinetic EOM in terms of standard Hamiltonian equation \eqref{1.1.3} so that
		\begin{equation}\label{1.2.3}
			\omega_{\alpha\beta}\dot{\xi}^\beta\equiv
			\left(\begin{array}{cc}
				-e \varepsilon_{ijk}B^k & \delta_{jk} \\
				-\delta_{jk} & \varepsilon_{ijk}\dfrac{\Omega^{k}}{\hbar}
			\end{array}\right)
			\left(\begin{array}{c}
				\dot{r}_j \\ \dot{p}_j
			\end{array}\right)=
			\left(\begin{array}{c}
				\partial_{p_j}\varepsilon \\ e E_j
			\end{array}\right)
			\equiv \partial_\alpha h.
		\end{equation}
		The Poisson brackets can be directly obtained from the inverse matrix.\par 
		As a bonus, we can also get the \emph{invariant volume element} for the symplectic manifold, which is defined to have a similar form on a manifold with metric structure
		\begin{equation}\label{1.2.4}
			\dd V:=\sqrt{|\mathrm{det}(\omega_{\alpha\beta})|}\prod_{\alpha=1}^{2d}\dd\xi^\alpha.
		\end{equation}
		Form \eqref{1.2.4} is clearly well-defined under local coordinate transformation (Jacobian cancels exactly). However, \textbf{one must be aware of the SHARP differences between symplectic manifolds and Riemannian manifolds. For example, there is no Darboux-like theorem for a Riemann manifold so we cannot locally trivialize a metric tensor by finding out the canonical variables, and there are infinite torsion-free metric compatible connection on symplectic manifolds (while on Riemannian manifold such connection is unique). See \url{https://physics.stackexchange.com/questions/136182} for more details}.\par
		Particularly in our cases, up to the first order of magnetic fields, the volume element takes the form of
		\begin{equation}\label{1.2.5}
			\dd V=\sqrt{1+\dfrac{2e}{\hbar}\bm{B}\cdot\bm{\Omega}}\,\dd\bm{r}\dd\bm{p}\simeq{\color{red}\left(1+\dfrac{e}{\hbar}\bm{B}\cdot\bm{\Omega}\right)}\dd\bm{r}\dd\bm{p}.
		\end{equation}

\section{Hydrodynamic EOM from Boltzmann Kinetic Equation}

\bibliography{hxd}
\bibliographystyle{apsrev} % apsrev is format for PRL of APS
\end{document}