\documentclass[10pt,nofootinbib,letterpaper]{revtex4}
\usepackage[nocap]{ctex}

%\usepackage{xeCJK}
%\setCJKmainfont{Source Han Sans CN}
%\setCJKmonofont{Source Han Sans CN}
%\setCJKsansfont{Source Han Sans CN}

\usepackage{amsmath,amssymb,amsfonts,mathrsfs,bm,dsfont}
\usepackage{enumerate}
\usepackage{enumitem} % Customize itemize, see https://ctan.org/tex-archive/macros/latex/contrib/enumitem/
\usepackage[all]{xy}
\usepackage[normalem]{ulem}	% delete line
\usepackage{graphics,color}
\usepackage{tikz}
	\usetikzlibrary{calc}
	\usetikzlibrary{decorations.markings}
	\usetikzlibrary{arrows}
	\usetikzlibrary{patterns}
	%\usetikzlibrary{shapes.callouts}
\tikzset{
    level/.style = {
        ultra thick,
        blue,
    },
    connect/.style = {
        dashed,
        red
    },
    label/.style = {
        text width=2cm
    }
}
\usepackage{pgfplots}
%\usepackage[citestyle=authortitle]{biblatex} % able to cite the title, author and year
%\usepackage{hyperref}
\usepackage{feynmp} % feymann diagram
\usepackage{extarrows}

\usepackage[normalem]{ulem} % 文字划掉(横),与 cite 兼容问题,见 https://tex.stackexchange.com/questions/98222/ulem-incompatibility-with-multiple-entries-in-cite

\newcommand*\dd{\mathop{}\!\mathrm{d}}
\newcounter{Claim}[section]
\newenvironment{Claim}[1][]{{\par\normalfont\bfseries \underline{Claim~\stepcounter{Claim}\arabic{Claim}.}~#1~~}}{\par}
\newcounter{Proposition}[section]
\newenvironment{Proposition}[1][]{{\par\normalfont\bfseries \underline{Proposition~\stepcounter{Proposition}\arabic{Proposition}.}~#1~~}}{\par}
\newcounter{Note}[section]
\newenvironment{Note}[1][]{{\par\normalfont\bfseries \underline{Note~\stepcounter{Note}\arabic{Note}.}~#1~~}}{\par}
\newcounter{Lemma}[section]
\newenvironment{Lemma}[1][]{{\par\normalfont\bfseries \underline{Lemma~\stepcounter{Lemma}\arabic{Lemma}.}~#1~~}}{\par}
\newcounter{Corollary}[section]
\newenvironment{Corollary}[1][]{{\par\normalfont\bfseries \underline{Corollary~\stepcounter{Corollary}\arabic{Corollary}.}~#1~~}}{\par}
\newenvironment{Proof}{{\par~{\normalfont\bfseries $\vartriangleright$}~~}}{\hfill $\square$\par\hfill\par} %\par
\newcounter{Def}[section]
\newenvironment{Def}[1][]{{\par\normalfont\bfseries \underline{Definition~\stepcounter{Def}\arabic{Def}.}~#1~~}}{\par}
\newcounter{Assumption}[section]
\newenvironment{Assumption}[1][]{{\par\normalfont\bfseries \underline{Assumption~\stepcounter{Assumption}\arabic{Assumption}.}~#1~~}}{\par}



\allowdisplaybreaks[4] %允许 align 跨页编排


%\def\checkmark{\tikz\fill[scale=0.4](0,.35) -- (.25,0) -- (1,.7) -- (.25,.15) -- cycle;}
%\def\G{\mathcal{G}}
\def\Z{\mathcal{Z}}
\def\H{\mathcal{H}}

\begin{document}
\title{Fiber bundles, Guage Transformation, and Chern-Simons Fields}
\author{Xiaodong Hu}
%\altaffiliation[Also at ]{Boson College}
\email{xiaodong.hu@bc.edu}
\affiliation{Department of Physics, Boston College}

\date{\today}

\begin{abstract}
    We review the memory matrix formalism with the example of parity-preserving transport, then switching to the parity-violating case (time-reversal symmetry is preserved) where anomalous Hall effect is expected to emerge. Such terms have already been revealed from pure hydrodynamic analysis in high energy physics. As another independent tool, memory matrix formalism is believe to provide the same result for the overlapping regime with hydrodynamics. In this letter, we will show the accodance of them.\par
    %\begin{center}
        \hfill\par
        {\centering\kaishu 流成笔下春风瓣,吹散弦上秋草声。\\[0.5em]}
    %\end{center}
    \hfill------ 雨楼清歌「一瓣河川」
\end{abstract}


\maketitle
\tableofcontents

\section{A Rush Course on Fiber Bundles}
    \subsection{Fiber Bundles, Principal Bundles, and Associated Bundles}
        The best language to interpret the quantum field theory is fiber bundles. We begine with the most general cases:
        \begin{Def}[(Differentiable Fiber Bundle)]
            A \emph{differentiable fiber bundle} is a tuple $(E,\pi,M,F)$ of three differentiable manifolds $E,M$ and $F$ and a smooth and surjective projection $\pi:E\rightarrow M$ such that
            \par1) For any $p\in M$, the inverse image $\pi^{-1}(\{p\})=F_{p}\subset E$ is diffeomorphic to $F$.
            \par2) For any $p\in M$, there exists a neighborhood $U_{i}$ of $p$ and a diffeomorphism $\phi_{i}:\pi^{-1}(U_{i})\rightarrow U_{i}\times F$ called \emph{local trivialization} such that the diagram
            %%%%%%%%%%%%% commute diagram %%%%%%%%%%%%%
            $$\xymatrix@C=1.5cm @R=1.5cm{U_{i}\times F \ar[d]_{\pi} & \pi^{-1}(U_{i}) \ar[l]_{\phi_{i}} \ar[dl]^{\pi} \\ U_{i} & }$$ 
            %%%%%%%%%%%%%%%%%%%%%%%%%%%%%%%%%%%%%%%%%%%
            commutes. $M$ is called the \emph{base space}, $E$ is called the \emph{total space} and $F$ is called the \emph{typical fiber}.
        \end{Def}

        For each non-empty intersection of two neighborhoods $U_{ij}=U_i\bigcap U_j$ and $p\in U_{ij}$, we have two local trivializations $\phi_i$ and $\phi_j$ giving any point of the pre-image $v\in\pi^{-1}(\{p\})$ two coordinate representations
        \begin{equation*}
            \phi_i(v)=(p,f_i),\quad \phi_j(v)=(p,f_j).
        \end{equation*}
        As elements on the typical fiber $F$, $f_i$ and $f_j$ are naturally connected as a {\color{red}\emph{left}\footnote{Here left action is consistent with the traditional convention because as an operator the automorphism $\mathop{\mathrm{Aut}}(F)$ is always assumed to act on $F$ on the left. But in principle you can define it as a right action as well. There is no other structure conflicting with this.} action $f_i=g_{ij} f_j$} by introducing the diffeomorphic \emph{transition function} $g_{ij}:U_{ij}\rightarrow\mathop{\mathrm{Aut}}(F), g_{ij}(\{p\})\equiv\phi_{i,p}\circ\phi_{j,p}^{-1}$, where $\phi_{i,p}^{-1}(f_{i})\equiv\phi_{i}^{-1}(p,f_{i})$. Clearly $g_{ij}$ satisfies the properties
        \begin{align*}
            g_{ii}&=\mathrm{id},\\
            g_{ij}&=g_{ji}^{-1},\\
            g_{ij}\cdot g_{jk}&=g_{ik},
        \end{align*}
        so forms a \emph{diffeomorephism group} $\mathrm{Diff}(F)$. But this goup is too large that we cannot gain any information about the fiber bundle\footnote{For example, to study a general linear space we should use group $\mathrm{GL}(n;\mathbb{C})$, but to study the inner product of a space we should work with unitary group $\mathrm{U}(n)\subset\mathrm{GL}(n;\mathbb{C})$.}. Thus to have a richer structure, we prefer to limit the domain of transition functions to a smaller subgroup as following.
        \begin{Note}
            It is local trivialization that determines the fiber bundle. In fact, given the bundle atlas $\mathcal{A}=\{(U_\alpha,\phi_\alpha)\}$ with $U_\alpha$ the open covering of $M$ and $\phi_\alpha$ the diffeomorphism, and assign the fiber space $F$, we can reconstruct the fiber bundle as following:
            \begin{itemize}
                \item Bundle space $\widetilde{E}=\displaystyle\coprod_\alpha U_\alpha\times F/\sim$, where $(x_\alpha\in U_\alpha,f_\alpha)\sim(x_\beta\in U_\beta,f_\beta)$ if $x_\alpha=x_\beta\in U_\alpha\bigcap U_\beta$ and $f_\alpha=g_{\alpha\beta}(x_\alpha)f_\beta$. Here $g_{\alpha\beta}\equiv\phi_\alpha\circ\phi_\beta^{-1}$ is the transition function.
                \item Projection is trivial $\widetilde{\pi}:E\rightarrow M, (x_\alpha,f)\mapsto x_\alpha$.
                \item Diffeomorphism that satisfies the universal property is also trivial $\widetilde{\phi}_\alpha:\widetilde{\pi}^{-1}(U_\alpha)\rightarrow U_\alpha\times F: (x_\alpha,f)\mapsto (x_\alpha,f)$.
            \end{itemize}
        \end{Note}
        \begin{Def}[(Fiber Bundle with Structure Group)]
            A differentiable fiber bundle $(E,\pi,M,F)$ is said to equip with a \emph{structure group} $G$ if it admits a bundle atlas $\mathcal{A}^G=\{(U_\alpha,\phi_\alpha)\}$ such that the transition function is defined by the {\color{red}\emph{left group action}} $g_{\alpha\beta}: U_{\alpha\beta}\rightarrow G\subset\mathop{\mathrm{Aut}}(F)$ 
        \end{Def}
        \begin{Example}
            A \emph{vecter} or \emph{tensor bundle} is the differentiable bundle in which $F=V$ or $\displaystyle F=\mathop{\bigotimes}_{i}V_{i}$. The structure group is natural to take value in $\mathop{\mathrm{Aut}}(F)=\mathrm{GL}(n)$. And a \emph{tagent bundle} $TM$ or \emph{cotangent bundle} $T^*M$ is the vector bundle where the vector space is the tagent (cotangent) space.
        \end{Example}

        \begin{Def}[(Principal G-bundle)]
            A \emph{principal G-bundle} $(E,\pi,M,F,G)$ is a differentiable fiber bundle with structure group $G$ together with a continuous {\color{red}right action} on the bundle $E\times G\rightarrow E, (v,g)\mapsto v\cdot g$  preserving the fibers $\pi(v\cdot g)=\pi(v)$ and acting \emph{freely}\footnote{By free we mean $\forall x\in F$ and $g,h\in G$, $x\cdot g=x\cdot h\implies g=h$.} and \emph{transitively}\footnote{By transitive we mean $\forall x,y\in F, \exists g\in G$ s.t. $x=y\cdot g$.} on them.
        \end{Def}
        \begin{Note}
            Unlike the cases before, since we have fixed the action of transtion function on the left as a convention, there is no other choice except the right action to place the newly defined action of the structure group. This is because for a compatible structure these two action have to commute with each other $g\circ g_{ij}=g_{ij}\circ g, (x,f)\mapsto(x,g_{ij}(x)\cdot f\cdot g)$, while in general they DO NOT if acting on the same side. See \url{https://mathoverflow.net/questions/50473} for more discussions.
        \end{Note}
        \begin{Note}
            Such free and transitive left group action gaurantes that the mapping $G\rightarrow F$ is bijective. So $F\cong G$ and equivalently we can delete the fiber space $F$ in our definition and replace it with the structure group $G$ for simplicity. This is why common literature denote the principal G-bundle as $E(M,G)$ and says ``\textbf{principal G-bundle is the differentiable fiber bundle with the fiber space coincides with the structure group $G$}''.
        \end{Note}
            
        \begin{Example}
            A \emph{frame bundle} is constructed by assigning each point $p\in M$ an \emph{ordered} basis of the same linear space (for example, tagent space $T_xM$). The set of all ordered frames $F_x$ admits naturally a free and transitive right action by $GL(n)$ (following from the standard linear algebra result that there is a unique invertible linear transformation sending one basis onto another). So \textbf{frame bundle is a principal bundle}. As a contrast, for vector (or tensor) bundle, since every vector space (or tensor space) contains the null vector (tensor) $\bm{0}$, the action of the structure group $GL(n)$ or its subgroups can never be free nor transitive. Namely, \textbf{vector (tensor) bundle is NOT a principal bundle}.
        \end{Example}
        
        \begin{Def}[(Section)]
            Given a fiber bundle $(E,\pi,M,F,G)$, suppose $U$ is one open set of $M$, then the smooth function $s:U\rightarrow E$ is called a \emph{local section}, if
            $$\pi\circ s=\mathrm{\mathrm{id}}_{M}.$$
        \end{Def}
        \begin{Theorem}
            \textbf{A principal bundle is trivial if and only if it admits a global section}. As a contrast, vector bundle always admits a global zero section by locally mapping every element $p\in U$ to the zero vector of $\pi^{-1}(\{p\})$. 
        \end{Theorem}

        \begin{Def}[(Associated Bundle of a Principal Bundle)]
            Let $P(M,G)$ be a principle G-bundle and $F$ a differentiable manifold on which the structure group $G$ has a continuous action as representation $\rho:G\rightarrow \mathrm{Homeo}(F)$ on the \emph{left}. Then we can construct a associated fiber bundle $Q=P\times_\rho F$ with $G$ the structrue group and $F$ the typical fiber as following:\par
            Utilizing the right action of $G$ on $P$ (as group multiplication) and left action on $F$ (as representation), we can define a \emph{left} action on the product manifold $P\times F$ by $g(u,f)\mapsto(u\cdot g,\rho(g^{-1})f)$. Then the quotient space $Q=P\times F/G$ is defined by assigning the equivalent relation $(u,f)\sim(u\cdot g,\rho(g^{-1})f)$. The surjective projection $\pi_Q:Q\rightarrow M$ is naturally induced by the canonical projection and the surjective projection of the original principal bundle $\pi:P\rightarrow M$ as following
            \begin{equation*}
                \pi_Q: \xymatrix{Q=P\times F/G \ar[r]^-\tau & P\times F \ar[r]^-\sigma & P \ar[r]^\pi & M}
            \end{equation*}
            So $Q=P\times F/G$ furnishes as a bundle space over $M$ with typical fiber $F$ and structural group $G$, called the \emph{associated fiber bundle} of the principal bundle.
        \end{Def}
        \begin{Note}
            The diffeomorphic local trivialization $\psi_\alpha$ of $(Q,\pi_Q,M,F)$ can also be induced by that of principal bundle as the commutative diagram
            \begin{equation*}
                \xymatrix@C=1.5cm @R=1.5cm{ & \pi_Q^{-1}(U_\alpha) \ar[ld]_{\psi_\alpha} \ar[d]^{\tau\circ\sigma} \ar@/^4pc/[dd]^{\pi_Q}\\ U_\alpha\times F \ar[dr]_{\pi,\pi_Q} & \pi^{-1}(U_\alpha)\subset P \ar[l]_{\phi_\alpha} \ar[d]^\pi \\ & U_\alpha}
            \end{equation*}
            so for $x\in U_\alpha$, the pre-image $\pi_Q^{-1}(\{x\in M\})=(x,f)$ is diffeomorphic to the fiber $F$, and we indeed define a fiber bundle.
        \end{Note}{
                \begin{Note}
                    The reverse procedure reconstructing a principal $G$ bundle from a fiber bundle with a $G$ structure can also be done as following:\par
                    On each point $x\in U_\alpha$, we have a left group action on the pre-image $\pi^{-1}(\{x\})\cong F$
        \end{Note}


    \subsection{Connection and Curvature on Fiber Bundles}

    
\section{Chern-Simons Field}


\end{document}