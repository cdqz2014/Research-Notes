\documentclass[10pt,nofootinbib]{revtex4}
%\usepackage{ctex}
\usepackage{amsmath,amssymb,amsfonts,mathrsfs,bm,dsfont}
\usepackage{graphics,color}
%\usepackage{hyperref}

\newcommand*\dd{\mathop{}\!\mathrm{d}}
\newcounter{Claim}[section]
\newenvironment{Claim}[1][]{{\par\normalfont\bfseries \underline{Claim~\stepcounter{Claim}\arabic{Claim}.}~#1~~}}{\par}
\newcounter{Proposition}[section]
\newenvironment{Proposition}[1][]{{\par\normalfont\bfseries \underline{Proposition~\stepcounter{Proposition}\arabic{Proposition}.}~#1~~}}{\par}
\newcounter{Note}[section]
\newenvironment{Note}[1][]{{\par\normalfont\bfseries \underline{Note~\stepcounter{Note}\arabic{Note}.}~#1~~}}{\par}
\newcounter{Lemma}[section]
\newenvironment{Lemma}[1][]{{\par\normalfont\bfseries \underline{Lemma~\stepcounter{Lemma}\arabic{Lemma}.}~#1~~}}{\par}
\newcounter{Corollary}[section]
\newenvironment{Corollary}[1][]{{\par\normalfont\bfseries \underline{Corollary~\stepcounter{Corollary}\arabic{Corollary}.}~#1~~}}{\par}
\newenvironment{Proof}{{\par~{\normalfont\bfseries $\vartriangleright$}~~}}{\hfill $\square$\par\hfill\par} %\par
\newcounter{Def}[section]
\newenvironment{Def}[1][]{{\par\normalfont\bfseries \underline{Definition~\stepcounter{Def}\arabic{Def}.}~#1~~}}{\par}


\def\Re{\mathop{\mathcal{R}e}}
\def\Im{\mathop{\mathcal{I}m}}

\numberwithin{equation}{section}

\begin{document}
\title{Duality in Condensed Matter Physics---From Lattice Models to Particle-Vortex Duality and Beyond}% Force line breaks with \\
%\thanks{This is a reminiscent note for Hubbard-Stratonovich Transformation.}%

\author{Xiaodong Hu}
%\altaffiliation[Also at ]{Boson College}
\email{xiaodong.hu@bc.edu}
\affiliation{Department of Physics, Boston College, MA 02135, USA}

\date{\today}


\begin{abstract}
	This is a research note documenting duality in condensed matter physics.
\end{abstract}
\maketitle
\tableofcontents
\section{Warm-up: Classical and Quantum Ising Duality}
	\subsection{$d=0$ Quantum Ising Model}
		$d=0$ Quantum Ising Model\footnote{This is pure \emph{statistical model} so has \emph{no} time dimensionality.} is special because there is no interactive term in Hamiltonian
		\begin{equation}\label{1.1.1}
			H_{q}=-g\sum_i\sigma_i^x.
		\end{equation}
		To compute the partition function of \eqref{1.1.1}, we need to slice the temperature (imaginary time) to $N$ segments and let $N$ tends to infinity. Namely,
		\begin{align}\label{1.1.2}
			\mathcal{Z}_{q}&=\mathop{\mathrm{tr}}e^{-\beta H_q}=\lim_{N\rightarrow\infty}\sum_{s_1,\cdots,s_N}\langle s_N|e^{+\Delta\tau g\sigma^x_1}|s_1\rangle \langle s_1|e^{+\Delta\tau g\sigma_2^x}|s_2\rangle\cdots \langle s_{N-1}|e^{+\Delta\tau g\sigma^x_N}|s_N\rangle\nonumber\\
			&=\lim_{N\rightarrow\infty}\sum_{s_1,\cdots,s_N}\prod_{i=0}^N\langle s_i|1+\Delta\tau g\sigma_j^x|s_j\rangle,
		\end{align}
		where $s_0\equiv s_N$ (PBC is applied naturally). Equation \eqref{1.1.2} shares the same structure as \emph{transferring matrix} methods in $1$-d \emph{classical} Ising model. In fact, ansatzing
		\begin{equation*}
			\langle s_i|1+\Delta\tau g\sigma_j^x|s_j\rangle= \langle s_i|Ae^{B\sigma_i^z\sigma_j^z}|s_j\rangle
		\end{equation*}
		and let $s_i=s_j=1$ and $s_i=-s_j=1$, one can immediately show that
		\begin{equation}\label{1.1.3}
			A^2=\Delta \tau g,\quad e^{-2B}=\Delta\tau g.
		\end{equation}
		Therefore, partition function of $0d$-quantum Ising Model can be re-written as
		\begin{equation}\label{1.1.4}
			\mathcal{Z}_{q}=A^N\mathop{\mathrm{tr}\exp \left(-\beta H_{c}\right) },
		\end{equation}
		where
		\begin{equation}\label{1.1.5}
			H_{c}=B\sum_{\langle ij\rangle}\sigma_i^z\sigma_j^z
		\end{equation}
		is the Hamiltonian of $1d$ classical Ising model.
	\subsection{$d>1$ Quantum Ising Model}
		Hamiltonian of $d>1$ Quantum Ising Model is
		\begin{equation}\label{1.2.1}
			H_q=-J\sum_{\langle ij\rangle}\sigma_i^z\sigma_j^z-g\sum_i\sigma_i^x.
		\end{equation}
		Still clues of the duality theory can be found from its partition function
		\begin{equation}\label{1.2.2}
			\langle s_i|e^{\Delta\tau(J\sum_{\langle ij\rangle}\sigma_i^z\sigma_j^z+g\sum_i\sigma_i^x)+\frac{Jg}{2!}\mathcal{O}(\Delta\tau^2)}|s_j\rangle=\sum_{s_k}\underbrace{\langle s_i|e^{\Delta\tau J\sum_{\langle ij\rangle}\sigma_i^z\sigma_j^z}|s_k\rangle}_{\text{d-dim Classical Ising Model}}\overbrace{\langle s_k|e^{\Delta\tau\sum_i\sigma_i^x}|s_j\rangle}^{\text{0-dim  Quantum Ising Model}},
		\end{equation}
		where B-C-H formula is utilized
		\begin{equation*}
			e^{A}e^B=e^{A+B+\frac12[A,B]+\frac{1}{12}[A,[A,B]]+\cdots}.
		\end{equation*}
		So we say \cite{hsieh2016d} {\color{red}\textbf{Partition function of d-dim quantum statistical model is equivalent to (d+1)-dim classical statistical model}}.

\section{$\mathbb{Z}_2$ Gauge Theory}
	$\mathbb{Z}_2$ gauge field describe the fluctuation of gauge freedoms, or visually fluctuation of loops on \emph{infinitely large} square lattice. Spins live on the links of sites, 
	\begin{equation}\label{2.1.1}
		H_{\mathbb{Z}_2}=-K\sum_\square\prod_{\ell\in\square}\sigma_\ell^z-g\sum_\ell\sigma_\ell^x.
	\end{equation}
	\subsection{Kramers-Wannier-Wegner Duality}
		Consider an \emph{infinite} square lattice, on the links of which spins are placed such that the Hamiltonian

\bibliography{hxd}
\bibliographystyle{apsrev} % apsrev is format for PRL of APS
\end{document}