\documentclass[10pt,nofootinbib]{revtex4}
\usepackage[nocap]{ctex}
\usepackage{amsmath,amssymb,amsfonts,mathrsfs,bm,dsfont}
\usepackage{enumerate}
\usepackage{enumitem} % Customize itemize, see https://ctan.org/tex-archive/macros/latex/contrib/enumitem/
\usepackage[all]{xy}
\usepackage[normalem]{ulem}	% delete line
\usepackage{graphics,color}
\usepackage{tikz}
	\usetikzlibrary{calc}
	\usetikzlibrary{decorations.markings}
	\usetikzlibrary{arrows}
	\usetikzlibrary{patterns}
	%\usetikzlibrary{shapes.callouts}
\tikzset{
    level/.style = {
        ultra thick,
        blue,
    },
    connect/.style = {
        dashed,
        red
    },
    label/.style = {
        text width=2cm
    }
}
\usepackage{pgfplots}
%\usepackage[citestyle=authortitle]{biblatex} % able to cite the title, author and year
%\usepackage{hyperref}
\usepackage{feynmp} % feymann diagram
\usepackage{extarrows}

\usepackage[normalem]{ulem} % 文字划掉(横),与 cite 兼容问题,见 https://tex.stackexchange.com/questions/98222/ulem-incompatibility-with-multiple-entries-in-cite

\newcommand*\dd{\mathop{}\!\mathrm{d}}
\newcounter{Claim}[section]
\newenvironment{Claim}[1][]{{\par\normalfont\bfseries \underline{Claim~\stepcounter{Claim}\arabic{Claim}.}~#1~~}}{\par}
\newcounter{Proposition}[section]
\newenvironment{Proposition}[1][]{{\par\normalfont\bfseries \underline{Proposition~\stepcounter{Proposition}\arabic{Proposition}.}~#1~~}}{\par}
\newcounter{Note}[section]
\newenvironment{Note}[1][]{{\par\normalfont\bfseries \underline{Note~\stepcounter{Note}\arabic{Note}.}~#1~~}}{\par}
\newcounter{Lemma}[section]
\newenvironment{Lemma}[1][]{{\par\normalfont\bfseries \underline{Lemma~\stepcounter{Lemma}\arabic{Lemma}.}~#1~~}}{\par}
\newcounter{Corollary}[section]
\newenvironment{Corollary}[1][]{{\par\normalfont\bfseries \underline{Corollary~\stepcounter{Corollary}\arabic{Corollary}.}~#1~~}}{\par}
\newenvironment{Proof}{{\par~{\normalfont\bfseries $\vartriangleright$}~~}}{\hfill $\square$\par\hfill\par} %\par
\newcounter{Def}[section]
\newenvironment{Def}[1][]{{\par\normalfont\bfseries \underline{Definition~\stepcounter{Def}\arabic{Def}.}~#1~~}}{\par}

\allowdisplaybreaks[4] %允许 align 跨页编排


\def\checkmark{\tikz\fill[scale=0.4](0,.35) -- (.25,0) -- (1,.7) -- (.25,.15) -- cycle;}


\begin{document}
\title{Algebraic Theory of Anyons}
\author{Xiaodong Hu}
%\altaffiliation[Also at ]{Boson College}
\email{xiaodong.hu@bc.edu}
\affiliation{Department of Physics, Boston College}

\date{\today}

\begin{abstract}
	This note plays a role as a supplementary material for the hidden but deep demanded mathematical backgrounds of the brilliant paper of Kitaev \cite{kitaev2006anyons}. Personal understanding and comments are inclued as well.\par
	%\begin{center}
		\hfill\par
		{\centering\kaishu 仰之弥高,钻之弥坚,瞻之在前,忽焉在后。\par}
	%\end{center}
	\hfill------ 「论语·子罕第九」
\end{abstract}

\maketitle
\tableofcontents

\section{Algebraic Theory of Anyons}
	\subsection{Category Theory: General Preliminaries}
		See my personal note of category theory. The missing Yoneda Lemma may also be extremely important for understanding the logic of emergentism in condensed matter physics.
	\subsection{Monoidal Categories}
		\begin{Def}[(Monoidal Category)]
			A monoidal category is a quintuple $(\mathcal{C},\otimes,\alpha,\mathds{1},\iota)$ with\par
			1) a category $\mathcal{C}$,\par
			2) a \emph{bifunctor} of \emph{product category} $\otimes:\mathcal{C}\times \mathcal{C}\rightarrow\mathcal{C}$ called \emph{tensor product},\par
			3) a \emph{natural isomorphism} $\alpha$ from the functor $(\bullet\otimes\bullet)\otimes\bullet:\mathcal{C}\times\mathcal{C}\times\mathcal{C}\rightarrow\mathcal{C}$ to the functor $\bullet\otimes(\bullet\otimes\bullet):\mathcal{C}\times\mathcal{C}\times\mathcal{C}\rightarrow\mathcal{C}$
			\begin{equation*}
				a_{XYZ}:(X\otimes Y)\otimes Z\mathop{\mapsto}^{\sim}X\otimes(Y\otimes Z)
			\end{equation*}
			called \emph{associaticity constraint}.\par
			4) an object $\mathds{1}\in\mathrm{Obj}(\mathcal{C})$,\par
			5) a \emph{natural isomorphism} $\iota$ from the functor $\mathds{1}\otimes\bullet:\mathcal{C}\rightarrow\mathcal{C}$ to the identical functor $id_{\mathcal{C}}:\mathcal{C}\rightarrow\mathcal{C}$ 
			\begin{equation*}
				\iota_X:\mathds{1}\otimes X\mathop{\mapsto}^\sim X
			\end{equation*}
			called \emph{unitality constraint}\footnote{We do not distinguish left and right unitality constraints as \cite{turaev2017monoidal} does.} such that the \emph{pentagon coherence}
			\begin{equation*}
				\xymatrix{&(X\otimes Y)\otimes (Z\otimes W)\ar[dr]^{\alpha_{X,Y,Z\otimes W}}&\\ ((X\otimes Y)\otimes Z)\otimes W\ar[ur]^{\alpha_{X\otimes Y,Z,W}}\ar[d]_{\alpha_{X,Y,Z}\otimes id_W}&&X\otimes(Y\otimes(Z\otimes W))\\ (X\otimes (Y\otimes Z))\otimes W\ar[rr]_{\alpha_{X,Y\otimes Z,W}}&&X\otimes((Y\otimes Z)\otimes W)\ar[u]_{id_X\otimes \alpha_{Y,Z,W}}}
			\end{equation*}
			and the \emph{triangle coherence}
			\begin{equation*}
				\xymatrix{& X\otimes Y\\ (X\otimes\mathds{1})\otimes Y\ar[ur]^{\iota_{X\otimes id_Y}}\ar[rr]_{\alpha_{X,id_{\mathds{1}},Y}} &&X\otimes(\mathds{1}\otimes Y)\ar[ul]_{id_X\otimes\iota_Y}}
			\end{equation*}
			commute.
		\end{Def}
	\subsection{Graphic Calculus}
		Drawn by hands.
	\subsection{Tensor Categories}

	\subsection{$6j$-Symbols}

\bibliography{hxd}
\bibliographystyle{apsrev} % apsrev is format for PRL of APS
\end{document}